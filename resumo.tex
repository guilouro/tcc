\begin{resumo}
	Atualmente o uso de meta-heur\'isticas
	vem sendo utilizada por sua simplicidade de implemente\c c\~ao e por abranger uma vasta 
	gama de problemas com os mais diversos n\'iveis de complexidade. 
	Neste trabalho estudamos a t\'ecnica de otimiza\c c\~ao denominada Otimiza\c c\~ao por 
	Enxame de Part\'iculas, Particle Swarm Optimization (PSO). 
	A qual, assim como ocorre em grande parte das meta-heur\'isticas, a qualidade 
	de aplica\c c\~ao dessa t\'ecnica deve definir alguns par\^ametros que influ\^enciam 
	diretamente no desempenho do algoritmo, como a in\'ercia que impede que uma 
	part\'icula mude de dire\c c\~ao instantaneamente, e a capacidade de aprendizado 
	coletivo e individual da part\'icula. Esses par\^ametros foram avaliados por um 
	conjunto de testes amplamente utilizado na literatura e com o objetivo de 
	observar como o algoritmo se comporta com cada grupo de par\^ametro em diferentes 
	situa\c c\^oes a ele apresentado. 
	
	Al\'em disso, um estudo mais amplo a respeito da t\'ecnica foi feito avaliando 
	algoritmos que adaptam seus par\^ametros no decorrer da execu\c c\~ao, buscando 
	assim uma ferramenta mais eficaz aplicada em diferentes classes de problemas. 
	Algoritmos que adaptam a in\'ercia por expemplo, geralmente permitem que a 
	part\'icula permanessa mais livre no in\'icio da execu\c c\~ao do programa e tornam-se 
	mais restritivas com o decorrer da busca. 
	
	As part\'iculas por sua vez, n\~ao dependem apenas da in\'ercia para evoluirem. Seu
	aprendizado se da por dois par\^ametros conhecidos como acelera\c c\~ao cognitiva e 
	aclera\c c\~ao social, que dizem o quanto uma part\'icula \'e influenciada para seguir o 
	aprendizado do enxame e o quanto ela deve buscar por si pr\'opria o ponto \'otimo.
	(Sendo assim, se uma execu\c c\~ao mais social, ou seja, que d\^e mais importancia para
	o aprendizado do enxame, pode levar ao fracasso de todo o grupo caso ele se perca.)
	Com isso, fica claro que a defini\c c\~ao desses par\^ametros compromete o resultado final da t\'ecnica.
	
	Al\'em desses tipos de adapta\c c\~ao, existem algoritimos que atualizam a 
	velocidade das part\'iculas de maneiras diferentes de acordo com o que as 
	part\'iculas aprenderam, bem como a inclus\~ao de novos par\^ametros necess\'arios 
	ao controle da adapt\c c\~ao.
	
	Um trabalho ainda mais elaborado foi o uso de 
	
	Portanto, este trabalho visa avaliar diversas propostas em torno da defini\c c\~ao 
	dos par\^ametros do PSO em um conjunto de problemas de otimiza\c c\~ao sem restri\c c\~oes, 
	visando identificar as propostas mais adequadas para diferentes classes de problemas.
\end{resumo}