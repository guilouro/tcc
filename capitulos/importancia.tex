\chapter{A import\^ancia de se usar Framework\label{cap:importancia}}
    Neste capítulo será apresentado, com dados técnicos, a importância do uso de um \emph{Framework} em projetos de desenvolvimento, detalhando algumas de suas vantagens e desvantagens no processo de codificação.

    O \emph{framework} é, como princ\'ipio b\'asico, uma arquitetura "padrão" que tem como objetivo fornecer ferramentas comuns a todo tipo de projeto, utilizando os mais variados tipos de Design Pattern (Padrões de Projeto) a fim de proporcionar um ambiente de desenvolvimento extremamente produtivo.

    Grande parte dos \emph{Frameworks} trabalham com um padrão principal denominado MVC (Model View Controller) que tem como base trabalhar com Modelo Lógico (Model), onde acontece toda a interação com a base de dados do projeto, Visualização (View), que é a parte responsável pela exibição de dados, e o Controle (Controller), que é a regra de negócios do projeto, pode-se dizer que o Controller é responsável por fazendo toda a comunicação com o Model e tratar os dados para serem exibidos pela View. Resumindo, o padrão MVC separa claramente o Design do Conteúdo e de sua Lógica.

    \section{Vantagens em usar um Framework\label{sec:vantagens-framework}}
        \begin{itemize}
            \item \textbf{Padronização em projetos}: A grande vantagem de um \emph{framework} é sua padronização no desenvolvimento. Por utilizar um conjunto já definido de Classes e Métodos, a necessidade de trabalhar conforme a ferramenta possibilita ajudar a garantir um aproveitamento maior de código projetos futuros.

            \item \textbf{Velocidade no desenvolvimento}: O fato de se fazer uso de módulos genéricos faz com que o \emph{framework} fique responsável por controlar o uso de funcionalidades repetitivas fazendo com que o desenvolvedor se concentre totalmente na regra de negócios de cada projeto.

            \item \textbf{Qualidade}: \emph{Frameworks} em geral são testados e atualizados a todo momento, tornado-se cada vez mais seguros e com melhores funcionalidades.

            \item \textbf{Re-uso de códigos}: A padronização de projetos torna capaz o re-uso de código sem dificuldades de adaptação.

            \item \textbf{Segurança}: Uma das vantagens mais importantes é a segurança que o \emph{Framework} pode dar ao projeto.

            \item \textbf{Fácil manutenção}: A separação do \emph{framework}, utilizando padrões de projetos, permite uma fácil manutenção em determinada ferramenta sem que afete outras.

            \item \textbf{Utilitários e Bibliotecas}: Classes e métodos embutidos no \emph{framework} para solucionar o problema de repetição contínua de códigos.
        \end{itemize}

    \section{Desvantagens em usar um Framework\label{sec:desvantagens-framework}}
        Esses pontos não são necessariamente uma desvantagem, porém são os principais motivos que inibem o desenvolvedor de iniciar em um \emph{Framework}.

        \begin{itemize}
            \item \textbf{Performance e peso}: A grande quantidade de arquivo, a chamada de métodos e a criação de objetos nem sempre necessários para determinados projetos tornam a aplicação pesada em alguns casos.

            \item \textbf{Curva de aprendizado}: Ao se trabalhar com códigos de terceiros, existe uma curva de aprendizado elevada e que fica dependente de uma boa documentação para conseguir atingir um bom ritmo de trabalho.

            \item \textbf{Conhecimento técnico}: É necessário que se tenha conhecimento em OOP (Programação Orientada a Objeto), boas práticas de programação e se entendam padrões de projetos para poder utilizar o \emph{Framework} da melhor forma.

        \end{itemize}
